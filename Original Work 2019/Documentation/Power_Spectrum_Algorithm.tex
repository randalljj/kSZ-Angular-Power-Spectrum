\documentclass[12pt]{article}

% Page formatting. 
\usepackage[a4paper,top=2cm,bottom=2cm,left=2cm,right=2cm,marginparwidth=1.75cm]{geometry}
\usepackage[parfill]{parskip}

% Typesetting for mathematics.
\usepackage{amsmath,amsthm,amssymb}
\usepackage{tikz-cd}

% Image Formatting
\usepackage{graphicx}
\usepackage{pdfpages}
\usepackage{subfig}

%============================================================================================================================
 
\begin{document}

\section{Power Spectrum Algorithm}

This is a detailed analysis of the algorithm used to calculate the integrand of the one halo power spectrum contribution given in equation (\ref{INT}). 
\begin{equation} \label{INT}
    P_{1h,Int}(M, k) = \frac{dn}{dM} [r_{s}(M)^3 \delta_a(M) \bar{u} (kr_s)]^2
\end{equation}
The function is initially called with the first argument being a 64-bit floating point number for the mass $M$ in units of $kg$ and the second argument being a 64-bit floating point number for the wavenumber $k$ in units of $Mpc^{-1}$. First, the concentration parameter $c(M)$ is called, given in equation (\ref{CP}), to reduce the number of function calls in following calculations. 
\begin{equation} \label{CP}
	c(M) = 
	\begin{cases}
		5\sigma(M) & M/M_0 < 10^{14} \\
		9\sigma(M) & M/M_0 \geq 10^{14} \\
	\end{cases}
\end{equation}
It is defined piece-wise such that for $M/M_\odot < 10^{14}$ then $5\sigma(M)$ is returned, else if $M/M_\odot \geq 10^{14}$ $9\sigma(M)$ is returned and stored as the variable $cp$. Next, the scale radius $r_s(M)$ is called and is given in equation (\ref{SR}).
\begin{equation} \label{SR}
    r_s(M) = \frac{1}{c(M)} \Big(\frac{3M}{800 \pi \rho_0} \Big)^{1/3}
\end{equation}
It takes the parameters $M$ and $cp$. The mean density of the universe $\rho_0$ has units of $kg / m^3$ so $r_s$ returns a value with units of meters and is stored in the variable $sr$. The next function we will observe is for $\delta_a(M)$ which is given in equation (\ref{DA}).
\begin{equation} \label{DA}
	\delta_a(M) = 
	\begin{cases}
		200c^{3} / 3[ln(1+c) - c/(1+c)] & M/M_0 < 10^{14} \\
		100c^{3} / ln(1+c^{3/2})        & M/M_0 \geq 10^{14}
	\end{cases}
\end{equation}
It takes parameters $M$ and $cp$. Similar to the function for the concentration parameter, if  $M/M_\odot < 10^{14}$ then the delta function for the NFW profile is returned, else if  $M/M_\odot \geq 10^{14}$ then the delta function for the Moore profile is returned. Next we will look at the function for the Fourier transform of the mixed density profile $\bar{\mu}(kr_s)$ found in equation (\ref{DP}). It uses the algebraic expressions found in Ma and Fry (2000). 
\begin{equation} \label{DP}
	\bar{\mu}(q) =
	\begin{cases}
		4\pi\{ln(e+1/q)-ln[ln(e+1/q)]/3\} / (1+q^{1.1})^{(2/1.1)} & M/M_0 < 10^{14} \\
		4\pi\{ln(e+1/q)+0.25ln[ln(e+1/q)]\} / (1+0.8q^{1.5})      & M/M_0 \geq 10^{14}
	\end{cases}
\end{equation}
It takes the parameters $k$, converted to $m^{-1}$, multiplied by $rs$ and $M$. Again, if $M/M_\odot < 10^{14}$ then the density profile for the NFW profile is returned, else if  $M/M_\odot \geq 10^{14}$ then the density profile for the Moore profile is returned. Then the results of those functions are multiplied together as shown in the parentheses and squared.

Next function is the Sheth-Tormen mass function $dn/dM$ found in equation (\ref{STM}).
\begin{equation} \label{STM}
	\frac{dn}{dM} = \frac{1}{M} \frac{dn}{dlnM} = \frac{\rho_{0}}{M} \frac{d ln \sigma^{-1}}{dM} 2A  \Big (1 + \frac{1}{\nu'^{2q}} \Big) \Big(\frac{\nu'^{2}}{2 \pi} \Big)^{1/2} e^{-\nu'^{2} / 2}
\end{equation}

Issues with differentiation of $\frac{d ln \sigma^{-1}}{dM}$ arose due to the limited precision in numerical calculations. I am using the method scipy.misc.derivative in the Scipy package for Python. It uses a third order finite difference method found in equation (\ref{FDM}).
\begin{equation} \label{FDM}
	f'(x) \approx \frac{f(x+dx) - f(x-dx)}{2dx}
\end{equation} 
It is based on the definition of the derivative through the limit process which requires the step size $dx \ll 1$. However, the point that I am taking the derivative at has a magnitude in the range $10^{40}-10^{50}$. Unfortunately, 64-bit floating point numbers have a precision of 15 significant figures. Therefore the addition and subtraction of numbers that have a greater difference than 15 orders of magnitude cannot be performed.
\end{document}













